\documentclass[]{article}
\usepackage{lmodern}
\usepackage{amssymb,amsmath}
\usepackage{ifxetex,ifluatex}
\usepackage{fixltx2e} % provides \textsubscript
\ifnum 0\ifxetex 1\fi\ifluatex 1\fi=0 % if pdftex
  \usepackage[T1]{fontenc}
  \usepackage[utf8]{inputenc}
\else % if luatex or xelatex
  \ifxetex
    \usepackage{mathspec}
  \else
    \usepackage{fontspec}
  \fi
  \defaultfontfeatures{Ligatures=TeX,Scale=MatchLowercase}
\fi
% use upquote if available, for straight quotes in verbatim environments
\IfFileExists{upquote.sty}{\usepackage{upquote}}{}
% use microtype if available
\IfFileExists{microtype.sty}{%
\usepackage{microtype}
\UseMicrotypeSet[protrusion]{basicmath} % disable protrusion for tt fonts
}{}
\usepackage[margin=1in]{geometry}
\usepackage{hyperref}
\hypersetup{unicode=true,
            pdfborder={0 0 0},
            breaklinks=true}
\urlstyle{same}  % don't use monospace font for urls
\usepackage{natbib}
\bibliographystyle{apalike}
\usepackage{longtable,booktabs}
\usepackage{graphicx,grffile}
\makeatletter
\def\maxwidth{\ifdim\Gin@nat@width>\linewidth\linewidth\else\Gin@nat@width\fi}
\def\maxheight{\ifdim\Gin@nat@height>\textheight\textheight\else\Gin@nat@height\fi}
\makeatother
% Scale images if necessary, so that they will not overflow the page
% margins by default, and it is still possible to overwrite the defaults
% using explicit options in \includegraphics[width, height, ...]{}
\setkeys{Gin}{width=\maxwidth,height=\maxheight,keepaspectratio}
\IfFileExists{parskip.sty}{%
\usepackage{parskip}
}{% else
\setlength{\parindent}{0pt}
\setlength{\parskip}{6pt plus 2pt minus 1pt}
}
\setlength{\emergencystretch}{3em}  % prevent overfull lines
\providecommand{\tightlist}{%
  \setlength{\itemsep}{0pt}\setlength{\parskip}{0pt}}
\setcounter{secnumdepth}{5}
% Redefines (sub)paragraphs to behave more like sections
\ifx\paragraph\undefined\else
\let\oldparagraph\paragraph
\renewcommand{\paragraph}[1]{\oldparagraph{#1}\mbox{}}
\fi
\ifx\subparagraph\undefined\else
\let\oldsubparagraph\subparagraph
\renewcommand{\subparagraph}[1]{\oldsubparagraph{#1}\mbox{}}
\fi

%%% Use protect on footnotes to avoid problems with footnotes in titles
\let\rmarkdownfootnote\footnote%
\def\footnote{\protect\rmarkdownfootnote}

%%% Change title format to be more compact
\usepackage{titling}

% Create subtitle command for use in maketitle
\newcommand{\subtitle}[1]{
  \posttitle{
    \begin{center}\large#1\end{center}
    }
}

\setlength{\droptitle}{-2em}
  \title{}
  \pretitle{\vspace{\droptitle}}
  \posttitle{}
  \author{}
  \preauthor{}\postauthor{}
  \date{}
  \predate{}\postdate{}

\usepackage{booktabs}
\usepackage{amsthm}
\usepackage{float}
\usepackage[utf8]{inputenc}
\usepackage[brazil]{babel}
\usepackage{eso-pic}
\newcommand\BackgroundPic{%
\put(0,0){%
\parbox[b][\paperheight]{\paperwidth}{%
\vfill
\centering
\includegraphics[width=\paperwidth,height=\paperheight,%
keepaspectratio]{ufpr.jpg}%
\vfill
}}}
\AddToShipoutPicture*{\BackgroundPic}

\begin{document}

\begin{titlepage}
\centering{\large{UNIVERSIDADE FEDERAL DO PARANÁ}}
\\
\centering{Departamento de Estatística}

\vspace{7.5cm}

\centering{\huge{OPERAÇÃO LAVA JATO TRI}}

\vspace{3cm}

CE095 - Teorias de Avalição

\vspace{2cm}

Andryas Waurzenczak, GRR: 20149125 \\
Gabriel Sartori, GRR: 2013xxxx


\vfill

07/06/2018
\end{titlepage}


\begin{abstract}
Abstract
\end{abstract}


\pagebreak
\tableofcontents
\pagebreak

O presente trabalho é fruto dos esforços da turma do 1º semestre de 2018
do curso de Teorias de Avaliação ministrada pelo professor
\href{}{Adilson dos Anjos}.

\section{Introdução}\label{introducao}

A Política é um tema bastante debatido nos mais diversos lugares, seja
nas universidades, bares, televisão, etc \ldots{} isto porque ela
interessa a todos nós. Dito isso, o presente trabalho é uma tentativa de
quantificar o quanto nossos amigos, amigos de nossos amigos, familiares
e pessoas ao nosso redor estão atualizados/informados sobre a política
atual do Brasil.

Para tal quantificação selecionamos um tema recente e que tem tido muita
repercusão. O assunto é a \textbf{Operaçao Lava Jato}, que é um conjunto
de investigações ainda em andamento pela Polícia Federal do Brasil, que
começou em 17 de março de 2014.

\section{Materias e Métodos}\label{materias-e-metodos}

O materias e métodos são descritos a seguir.

\subsection{Materias}\label{materias}

O conjunto de dados é um produto dos esforços da turma de Teorias de
Avaliação, 1º semestre de 2018, com uma pequena contribuição da turma
passada. A forma de coleta se deu atráves de um formulario online que
ficou disponivel na plataforma do
\href{https://www.google.com/forms/about/}{Google} por 21 dias.

O desenvolvimento do questinário foi feito em 5 etapas.

\begin{enumerate}
\def\labelenumi{\arabic{enumi}.}
\tightlist
\item
  Elaboração dos itens
\item
  Validação dos itens
\item
  Seleção dos itens
\item
  Elaboração de Fatores Associados
\item
  Disponibilização do formulário
\end{enumerate}

Para a execução da 1ª e 2ª etapa utilizou-se como embasamento o
\href{www.ufpr.br/~aanjos/CE095/guia_elaboracao_revisao_itens_2012_INEP.pdf}{guia
de elaboração de revisão de itens da INEP - 2012}. Cada aluno
desenvolveu 3 questões que foram depois distribuidas de forma aleatoria
para um dos colegas avaliar se o item estava de acordo ou não. A ideia
básica para a criação e validação dos itens era possuir
\textbf{TEXTO-BASE}, \textbf{ENUNCIADO}, \textbf{ALTERNATIVAS} e
\textbf{GABARITO}. Dos itens que passaram dessas 2 primeiras etapas, 20
foram selecionados e foram complementados com mais 6 itens de um
instrumento de medida anterior ao nosso que apresentaram boa calibração.
Ao todo tivemos 26 itens no nosso intrusmento de medida.

Após isso foi elaborado candidatos a fatores associados dos quais foi
escolhido três e então o questionario foi disponibilizado no dia 10 de
Maio de 2018.

\textbf{Descrição do conjunto de dados com summary}

\subsection{Métodos}\label{metodos}

Falar os métodos utilizados (descreve-los) Modelo de três parâmetros

\subsection{Recursos Computacionais}\label{recursos-computacionais}

Para as análises o software utilizado foi \citet{software-r} e os
pacotes utilizados foram:

\section{Resultados}\label{resultados}

\subsection{Análise Descritiva}\label{analise-descritiva}

Curva caracteristica do teste Curva dos itens mais relevantes dos itens
menos relevantes maior score menor score

\subsection{Aplicação do Modelo X}\label{aplicacao-do-modelo-x}

Criar escala interpretavel

\section{Considerações Finais}\label{consideracoes-finais}

\bibliography{book.bib}


\end{document}
